\documentclass[letter,12pt,oneside]{book}
\usepackage[T1]{fontenc}
\usepackage[utf8]{inputenc}
\usepackage{CJKutf8}
\usepackage{lmodern}
\usepackage{hyperref}
\usepackage{graphicx}
\usepackage{indentfirst}


\setlength{\parindent}{2em}

\linespread{1.3}
% \setlength{\parskip}{1ex}
\setlength{\parskip}{0.5\baselineskip}

\newenvironment{dedication}
{
   \clearpage
   \thispagestyle{empty}
   \vspace*{\stretch{2}}
   \hfill\begin{minipage}[t]{.7\textwidth}
   %\raggedright
}
{
   \end{minipage}
   \vspace*{\stretch{3}}
%    \clearpage
}

\begin{document}
\begin{CJK}{UTF8}{gbsn}
\title{\Huge \textbf{ 博士磨砺 } \\ \large \textbf{一个博士研究生的回忆录} \footnote{英文原文链接:\url{http://www.pgbovine.net/PhD-memoir/pguo-PhD-grind.pdf}}}
% Author
\author{{ \begin{tabular}{lll}
原著 & Philip J. Guo & (\texttt{philip@pgbovine.net})\\
翻译 & 齐鹏 & (\texttt{pengqi@cs.stanford.edu})
\end{tabular} }}
\date{}

\renewcommand*\contentsname{目录}  

\frontmatter
\maketitle

\begin{dedication}
献给所有热爱创造的人。
\end{dedication}

\tableofcontents

\mainmatter

\setcounter{chapter}{0}
\newcommand{\mychapter}[1]{
    \addtocounter{chapter}{1}
    \setcounter{section}{0}
    \chapter*{#1}
    \markboth{#1}{#1}
    \addcontentsline{toc}{chapter}{#1}
}

\renewcommand{\emph}[1]{{\CJKfamily{gkai}#1}}

\newcommand{\breakline}[0]{\begin{center}$\sim$\end{center}}

%%% Disclaimer
\markboth{}{}
\section*{Disclaimer}

This is an unauthorized Chinese translation of Philip J. Guo's memoir \emph{The Ph.D. Grind}, and the original author did not have any input on the translation.

The copyright of the content belongs to the original author, and the translation to the translator. This work may not be used for business purposes, and may only be used as non-commercial material.

Apart from the Translator's Preface, none of the content of this work represents, none should be interpreted as, the opinion of the translator.

\section*{声明}

本作品是Philip J. Guo的回忆录《The Ph.D. Grind》的中文翻译,翻译并没有得到原作者的任何授权,原作者也并没有以任何形式参与到翻译过程中。

本作品内容的版权归原作者所有,翻译版本归译者所有。本作品不得被用于任何商业目的,只能作为非盈利性材料传播。

除译序外,本作品的一切内容均不代表——且不应被认为是——译者的观点。

\clearpage

%%%% Preface

\chapter*{序}
\markboth{}{}

这本书记述了我在斯坦福大学从2006年到2012年,攻读博士研究生期间六年的求学经历。这本书适合广泛的读者群,其中包括:

\begin{itemize}
\item 有志攻读博士研究生\footnote{研究生(graduate student)本是指本科之后的高等教育,通常包括硕士研究生和博士研究生两种学位。现代汉语中“研究生”一词经常被滥用,用以单指硕士研究生,这其实是错误的。——译者注}的本科生;
\item 寻求方向或灵感的在读博士生;
\item 希望更深入了解博士研究生的教授;
\item 希望聘用和管理拥有博士学位员工的雇主;
\item 在充满竞争的创新领域工作、与自我追求和自我激励密不可分的专业人士;
\item 对学术研究充满好奇的有一定教育背景的成年人(或者早熟的青少年)。
\end{itemize}


《博士磨砺》与已有的与博士经历相关的文章在写作形式、写作时机和写作基调上都有所不同:

\paragraph{形式} 《博士磨砺》是一本面向大众的回忆录,而非一本面向在读博士生的“成功指南”。尽管博士生也能在我的经历中学习到经验和教训,但我的目标并不是直接提供建议。对于博士生而言,市面上的“成功指南”和“建议专栏”已经不胜枚举,我也无意画蛇添足。这些文章充满了“持之以恒”和“不积跬步,无以至千里”等等空泛的词汇,但相反,回忆录的形式让我能丰富、具体地叙述发生在我自己身上的故事。

\paragraph{时机} 《博士磨砺》是我在完成博士学位之后马上着手写作的,而这正是撰写这样一本回忆录的最佳时机。不同于在读博士生的是,我可以在回忆录中对整个博士求学期间的工作进行系统的整理和反思;而相比资历较深的研究者而言,我可能更容易忠于攻读期间的经历,不会引入过多由研究经历带来的有选择性的观点和感受。


\paragraph{基调} 尽管保持完全客观是不可能的,但我在写作《博士磨砺》过程中仍然试图贯穿一种相对客观的基调。与其他作者不同的是,很多撰写博士相关文章、书籍,或者绘制漫画的人通常属于以下两类之一:
\begin{itemize}
\item 成功的教授或者科学家,他们通常给出一些冠冕堂皇的建议,比如他们可能会说:“研究生生活诚然辛苦,但它同时应该是一段美好的知识之旅,你应该享受这个过程、并从中受益……因为我当年就是这样做的!”
\item 或者是苦涩的博士研究生或者辍学博士,他们常常因为自己的经历留下了心理阴影,当提及博士生活的时候会用一种过分夸张、“看破一切”、自我怨恨的腔调:“啊,那时我的世界就是个活生生的地狱,我到底拿我的青春换了什么?!?”
\end{itemize}

冠冕堂皇的建议可能能激励一些学生,而大倒苦水的呻吟可能能引起另一些处境不佳的学生的共鸣,但作为大众读者而言,他们可能并不能感受到这些极端的情绪。

最后,在我开始讲述自己的故事之前,我希望强调一下,每个博士研究生的经历与他/她所在的学校、院系、研究领域、经费状况等都有巨大的关系。在我的读博生涯中,我感到自己非常幸运,能很大程度上自由自愿地完成自己的学业;我知道很多学生相比而言受到了很多的限制。我的故事只是一个孤立的数据点,所以我所呈现的故事可能并不能泛化、推广到每一个人。然而,我会尽力避免叙述变得过分局限于我个人的情况。

祝阅读愉快!

\begin{flushright}
Philip Guo, 2012年6月
\end{flushright}

\mainmatter

%%% Prologue

\mychapter{前言}

因为我本科的专业是电子工程和计算机科学,大部分本科同学在毕业之后,都马上投入到了工程性的工作中了。而我最终选择攻读博士学位,究其原因,一方面是受到来自父母潜移默化的影响,另一方面也和我在本科期间对工程性的实习产生的不良印象不无关系。

我父母从未要求过我去攻读博士学位,但我可以看出,终身大学教授\footnote{在美国和加拿大等国家,终身职位(tenure)是指资深学者拥有的,除因正当理由外,免于被解雇的合同权利。与此不同的是任意性职位(at-will),规定劳动双方随时可以以任何理由终止劳动合同,不需要法定的正当理由,也不需要提前通知。大多数非学术研究类(工程性)的工作属于后者。——译者注}是他们最尊重的职业——而博士学位正是成为终身教授的必要条件。为什么终身教授会是他们心目中的理想职业呢?这其实并不是因为他们对纯粹的学术追求有什么不切实际的盲目推崇。尽管我的父母也都到过良好的教育,但他们同时也是非常现实的移民——一份终身教授的职位对他们而言,更大的吸引力在于终身制所带来的工作保障。

我父母的很多朋友都是在企业的工程性职位上供职的中国移民。由于他们在英语技巧和美国文化了解上的不足,他们中的大多数人在职业生涯中的经历都并不顺利,而这一问题往往随着年龄增长而更加突出。在假日聚会上,我经常能听到一个不变的主题:人们的工作因为难以相处的经理而处处不顺、成为年龄歧视和“玻璃天花板”效应\footnote{玻璃天花板(Glass Ceiling)效应是一个政治术语,用于形容“在企业内升职过程中看不到但难以逾越的障碍,通常见于少数人种和女性身上,且与这些人群的资历和成就并无关联”的现象。——译者注}的牺牲品、甚至面临大规模裁员和长期失业的风险。尽管我父亲不是一位工程师,而是在高技术产业就职,但他也难逃这种魔咒,在一系列和管理层极其官僚做派的斗争中失败,最终早早地结束了他在公司的工作。那时他还相对很年轻,只有45岁。

我的母亲则是这种不幸潮流中唯一的幸存者。她十分热爱自己在UCLA\footnote{UCLA是加州大学洛杉矶分校(University of California, Los Angeles)的简称。——译者注}作为社会学终身教授的工作。和她的大部分中国移民朋友不同的是,她享受终身职务保障,不用向老板汇报工作,可以几乎完全自由地追求她自己的学术兴趣,在她的研究领域也小有名气。亲眼目睹我母亲成功的职业轨迹,和父亲及他们的朋友们在职业生涯上的恶性循环两者之间的巨大反差,这在我高中和大学本科的学习生涯中留下了难以磨灭的印象。

当然,仅仅因为这种非理性的、年少时的恐惧就去选择读博显然是不明智的。为了让自己对在企业的工作生活有所印象,我在本科的每个假期都参加了工程性公司的实习。因为我工作过的办公室都碰巧只有我一个实习生,我被赋予了一种罕有的特权——我的工作职责是按照全职初级工程师的标准分配的。尽管在这个过程中我学到了很多技术上的技巧,我仍然觉得这种日复一日的工作极少需要思考且非常无聊;这也可能与我实习过的公司不是一流公司有关。我本科的很多朋友都在微软和谷歌等一流公司实习过,并非常喜欢他们的实习经历,他们最终也往往在毕业后和这些公司签下了全职工作合同。

因为我对我的实习经历感到厌倦,而另一方面对本科时作教学助理(助教)和研究助理(助研)的经历比较感兴趣,我当时将未来的职业目标定在了大学教学和学术研究上。等到我在MIT\footnote{MIT是麻省理工学院(Massachusetts Institute of Technology)的简称。——译者注}的第三年过半,我已经下定决心在毕业后攻读博士学位,因为这是实现我职业目标的必经之路。我决定留在MIT,完成一个五年制本硕连读的项目,以此来在申请博士项目之前积累更多的研究经验,以期能被录取到更多排名顶尖的院系中去。

我找到了一位硕士论文的导师,并且就像很多壮志踌躇的年轻人一样,开始向他阐述自己不甚成熟、遑论完善的研究思路和计划。我的导师很耐心的听我说完了我的想法,但最终仍然成功地说服我进行一些和他的研究兴趣更契合、更主流的研究,而且更重要的是,这些研究项目更契合他的基金项目要求。因为当时我的硕士项目学费一部分来自我的导师从美国政府申请的一个\emph{研究基金},我有义务在基金所规定的范围内完成研究工作。因此,我听从了他的建议,并将接下来两年半的时间用于开发一类新的原型工具,用于分析由C和C++语言编写的计算机程序的运行时行为。\footnote{运行时行为分析(run-time behavior analysis)是计算机程序编写和维护中一种重要的技术,它可以帮助软件工程师更快、更准确地发现在程序编写时难以注意到的程序缺陷和漏洞(bug),进而完善其功能并提高稳定性。——译者注}

虽然我并不是非常热衷于我硕士论文的项目,但事实证明选择一个和导师研究兴趣契合的项目是一个明智的选择:在他有力的指导下,我发表了两篇论文——一篇我被列为第一作者(主要作者),另一篇的位置稍微靠后——并且我的硕士论文获得了系里年度\emph{最佳毕业论文奖}。这些成就,加上导师在我申请文档中的帮助,为我赢得了几所顶尖计算机科学系博士项目的录取。因为斯坦福是我的首选,在我收到录取的当天晚上,我甚至激动得几乎无法入睡。

我还非常幸运地得到了NSF\footnote{NSF是美国国家自然基金会(National Science Foundation)的简称。NSF是支持美国各大高等院校进行基础自然科学研究的重要基金来源之一。该奖学金的申请只面向美国公民及永久居民开放。——译者注}和NDSEG\footnote{NDSEG是美国国防科学与工程研究生奖学金(National Defense Science \& Engineering  Graduate Fellowship)的简称。该奖学金由美国国防部出资设立,旨在推动国防相关的科学与工程研究的发展,其申请只面向拥有美国国籍的人士开放。——译者注}研究生奖学金的垂青,这两者都只颁发给了大概5\%的申请者。这两个奖学金为我免除了攻读博士的六年之中,五年的全部费用,也使我不必完成各种研究基金相关的研究项目。与我不同的是,在我的研究领域中,大部分博士研究生依赖于教授提供的研究经费和院系提供的助教经费。博士研究生的经费包括全额学费,以及大约每个月1,800美元的补贴,用于贴补生活开支。(在我的研究领域几乎没有人自费攻读博士,因为那样在经济上非常不划算。)

因为我已经有一定的研究和写作论文的经验,当我在2006年9月来到斯坦福时,我觉得自己已经为未来艰苦的博士研究做好了充足的准备。然而,那时我完全没有预见到的是,我的博士第一年即将成为我生命中到此为止最为打击信心、令人灰心丧气的一段时间。

%%% Year One: Downfall

\mychapter{第一年:失落}

2006年的夏天,在我开始在斯坦福攻读博士学位的几个月前,我考虑了一些我认为自己感兴趣研究的课题。大题而言,我想要创造一些创新性的工具,用来帮助人们在进行计算机编程时提高效率,换言之,\emph{提高程序员生产力}。我之所以对这个方向感兴趣,主要源于我在暑期实习中自己的编程经历:因为日复一日,公司分配给我的工作并不能让我提起太多兴趣,工作中的很多时间,我都坐在自己的格子间里反思,在我工作的这些公司中计算机编程的流程如何低效。那时我认为,如果能投身于旨在降低这种低效性的科学研究,应该是不错的方向。更宽泛地说,我的研究兴趣集中在能让所有类型的计算机使用者更加高效的工作中——而非仅仅聚焦在专业程序员的身上。举例来说,我希望能设计出新的工具,能帮助科学家分析和绘制数据、帮助系统管理员调整服务器配置、或者帮助计算机新手学习使用新的软件。

尽管我当时就有这些模糊、不成型的兴趣,但距离我将这些兴趣转化为真正可以发表的研究项目,并最终形成一篇\emph{博士论文},还有很多年的差距。对斯坦福计算机科学系的博士研究生而言,通常他们需要发表二到四篇第一作者的学术论文,并将这些论文合并成一篇博士论文——通常是一篇长度与书籍相仿的科技文档。当博士论文通过一个由三位教授组成的\emph{博士论文委员会}批准后,学生就可以毕业,从而获得博士学位了。在我所在的计算机系,通常一个博士研究生需要四到八年毕业,这个年限取决于他们发表文章的效率。

在2006年9月的新生信息会\footnote{信息会(Orientation),或译为迎新会,是在学校或大型组织中常见的,统一为新成员提供常用信息、帮助其尽快适应环境的活动,通常由资深成员主持。}上,系里的教授鼓励所有的博士新生尽快找到自己的\emph{导师},所以我和我的同学们一样,把一开始的几个月花在了找教授谈话上,希望尽快找到一个研究方向匹配的导师。对于一个学生的论文委员会来说,导师扮演了最重要的角色,因为他/她对学生能否毕业拥有最终的决定权。在我的研究领域,导师通常还负责通过自己的科研经费为学生提供经费支持,并且指导他们开展课题研究、写作论文。和几位教授谈话后,我发现Dawson\footnote{Dawson是这位教授的名字(first name),而非姓(last name),故翻译成Dawson教授是不准确的——原文也没有出现Professor Dawson(Dawson教授)这一称呼。单独使用名字在英语中非常普遍,通常用于熟人之间互相称呼、陌生人之间互相介绍、以及不希望提及全名的场合。文中出现的人名都是单独的名字。为了忠于原文的阅读体验,译者没有对这些名字进行标准化翻译和加工。——译者注}和我的研究兴趣和研究风格都似乎最为接近,所以我选择了他作为我的导师。

在我刚去斯坦福的时候,Dawson已经在斯坦福度过了八年,并且刚刚获得终身教授职位;通常,教授在他们工作的前七年如果发表了足够多高水平的研究论文,就可以得到\emph{终身职位}(终生的工作保障)。Dawson的主要研究兴趣是创造新的工具以自动在复杂、真实的软件中寻找\emph{bug}(软件代码中的缺陷)。在过去的十年间,Dawson和他的学生编写了许多这样的工具,相比他们的竞争者而言,他们能找到程序中更多的bug。他们的研究成果十分有效——他们甚至城里了一个成功的创业公司,通过提供基于这类技术的缺陷查找服务而盈利。尽管我对Dawson的研究项目感兴趣,更吸引我的一点则是他的研究哲学和我自己的想法十分接近:他是充满激情的“务实派”——相比于单纯为了显得“学术”而去研究理论上“新颖”的课题,他更关注的是得到实实在在、有说服力的结果。

我和Dawson第一次面谈时,他似乎对我的大方向——让计算机的使用和计算机编程变得更高效——只是稍感兴趣。不过,有一点他说的很清楚:他非常希望招收一些新的研究生来帮助他完成一个叫\emph{Klee}的软件缺陷检查工具,因为这个项目是他现在的科研经费所支持的。(这个工具有好几个名字,但为了简便,这里就叫它“Klee”。)和其他的教授和高年级博士谈过之后我才意识到,对于新生而言,加入一个已有的、由研究基金支持的项目是一种常规现象,而并不是马上开始进行自己原创的研究项目。我说服了自己,认为自动查找软件缺陷也是一种间接提高程序员工作效率的研究,于是便打定主意,加入Klee项目组。

当2006年12月,我准备加入Klee项目组时,Dawson已经在指导五个学生参加这个项目了。项目组的带头人Cristi是第三年的博士生,而正是他和Dawson开发了最初版本的Klee。Dawson、Cristi和其他研究者不久前还发表了第一篇说明Klee系统的论文,并展示了Klee在发现一些新的缺陷上十分有效。那篇论文受到了学术界的好评,所以Dawson希望保持这一势头,继续发表几篇跟进这一项目的文章。值得注意的是,从同一个研究项目中发表多篇论文是可能的(即“跟进论文”),只要这些新的论文有新的原创观点,相比前作的改进和创新,或者相比前作而言在结果上有很大的改善。当时,下一个相关领域的\emph{顶级会议}的论文提交截止日期是2007年3月,所以Klee团队有四个月用于基于之前的论文做出创新性的改进,以期发表一篇新论文。

\breakline

在我继续讲述我的故事之前,我想简单地介绍一下学术论文是如何评审和发表的。在计算机科学领域,发表文章最受关注的场合是\emph{学术会议}。当然,值得指出的是,再很多其他学科中,\emph{学术期刊}才是最受关注的,而对这些领域而言,“学术会议”往往和计算机科学领域大相径庭。对计算机科学而言,学术会议的论文发表流程大概如下:

\begin{enumerate}
\item 每个会议发布一个征稿启事,其中说明了会议所要求的课题范围和一个论文提交的截止日期。
\item 研究者需要在指定的截止日期之前提交自己的论文。通常每个学术会议会收到100到300份论文稿件,每篇论文大概包含30到40页双倍行距\footnote{双倍行距实质行间距与文字高度相同,结合单栏排版,通常用于学术期刊初稿的排版。计算机科学的学术论文通常采用单倍行距、双栏排版。——译者注}的文字。
\item 学术会议的程序委员会(PC)\footnote{PC是程序委员会(Program Committee)的简称。——译者注}通常由大约20位专家研究者组成,他们负责将论文负责分类,以便审稿。每篇论文通常由三到五个人完成评审,参与评审的人员可能包括PC的成员,或者由PC成员邀请的、来自学术界自愿参与审稿过程的审稿人。论文的评审过程通常需要大约三个月。
\item 当每位PC成员都完成审稿后,整个委员会将开会商议,通过审稿人的反馈决定接收一部分论文稿件,并拒收剩下的稿件。
\item 程序委员会会发出电子邮件通知所有的作者,告知他们论文是否被接收,并将审稿人对他们的论文提出的审稿评价附在电子邮件中。
\item 论文被接收的作者参加学术会议,并关于自己的论文做一个30分钟左右的演讲。学术会议结束后,所有的论文都将被收录在在线的数字图书馆中。\footnote{根据研究领域不同,学术会议的举办方式略有不同。在一些领域中,论文会被PC分为演讲展示(oral presentation)和海报展示(poster presentation)两种形式。演讲展示中,通常有5至30分钟供作者在报告厅等场合公开真实自己的研究成果;海报展示时,所有参与的作者则将自己的研究成果展示在一张海报上,并以海报为基础向观众展示自己的研究。通常演讲展示会有更多听众和更大的影响力,在有演讲展示和海报展示之分的学术会议中,PC也会把有限的演讲展示机会分配给他们认为更有影响力的学术研究项目。——译者注}
\end{enumerate}

通常,一个备受关注的\emph{顶级}学术会议的论文接收率在8\%到16\%之间,而\emph{第二级}的学术会议大概接收20\%到30\%的投稿。由于这些接收率相对较低,对于一篇学术论文来说,被拒收、修改并重新提交并不罕见——在论文最终被接收之前可能这一过程要重复多次,而这一过程可能会花费数年的时间。(在同一时间,一篇论文只能被提交到一个会议。)

\breakline

当Dawson说他想要提交一篇论文到2007年3月截止提交的顶级会议之后,他向我介绍了当时其他五个学生工作的方向,并让我选择自己感兴趣的工作。我选择了使用Klee来寻找\emph{Linux驱动程序}中存在的新的程序缺陷。\emph{驱动程序}是指用于帮助操作系统完成其与外置设备(如鼠标或键盘)通信的软件代码。而\emph{Linux操作系统},和微软Windows系统或者苹果Mac OS系统类似,包含了成千上万这样的驱动程序,用以连接各种各样的外置设备。对于传统的调试方法而言,驱动程序中的软件缺陷十分难以找到,甚至有可能是危险的,因为驱动程序的缺陷可能会导致操作系统死机甚至崩溃。

Dawson认为Klee可以在Linux驱动程序的成千上万行代码中,找到其他自动缺陷分析软件(甚至人)从未找到过的软件缺陷。我记得我当时考虑过,尽管在Linux驱动程序中找到缺陷写在论文里看起来很不错,但我并不是很明白这样的工作能不能算是真正的研究贡献。按照我的理解,我要做的事情是用Klee去寻找程序缺陷——将一个已有的研究应用在实际问题上——而不是采用一种创新的方法来提高Klee的性能。此外,我并不明白,到三月份论文截稿时,我的工作和其他五个学生的工作如何能融合成一篇自洽的论文。尽管如此,我当时相信Dawson在头脑中有一个高瞻远瞩的思路完成这篇论文。介于我刚刚加入研究项目,我并不想马上开始对这些应该由教授决定的问题开始指手画脚。任务已经摆在眼前,我要考虑的只是尽我所能,完成这个任务。

\breakline

我博士生涯的前四个月被我用于配置Klee以用它来分析上千行的Linux设备驱动程序,以期从中发现新的程序缺陷——这一过程并不顺利。尽管看起来我的任务并不复杂,但我很快就被淹没在了一些细节问题中,而这些细节对于让Klee能够分析Linux驱动程序又是必不可少的。我经常会花几个小时设置好Klee所需的复杂的实验环境,以分析某一个驱动程序的程序缺陷;但这种工作又往往以Klee因为其自身的程序缺陷而崩溃告终,让我的努力付诸东流。当我把这些Klee中存在的问题报告给Cristi时,他会竭尽全力解决,但由于Klee本身极其复杂、环节众多,在其中找到并解决程序缺陷绝非易事。我并不是想要专门指责Klee:任何以科研为目的开发的原型软件都会或多或少有一些难以预见的缺陷。我的任务是用Klee去Linux驱动程序代码中寻找缺陷,但讽刺的是,我整个第一个月的工作都变成了在Klee里找缺陷。(Klee不能在自己的代码里自动找到缺陷,这实在是太遗憾了!)随着时间的流逝,我对于手头的工作越来越感到沮丧,我觉得自己被分配的任务就是单纯的苦力,毫无知识含量——我的时间都用在让Klee能正常运行上了。

这是我生命中第一次感受到被手头的工作所淹没的无望。以前,我的暑期实习都相对不难,而且就算一些学校的作业对我来说有些挑战,作业中也总有一个需要找到的正确答案等着我。如果我课上有没听懂的内容,助教或者高年级的学生总可以帮我答疑解惑。就算在本科进行学术研究的时候,我也总能请辅导我的博士学长帮忙,因为我当时处理的问题相对比较简单,而他通常知道问题的解决方法。对于本科的研究助理而言,对学术研究的投入以及期望也相对较低:科研只是我日常生活很小的一部分。如果我在某个科研问题上毫无头绪,我可以选择集中精力在课程作业上,或者干脆和朋友们出去玩。本科毕业也和学术研究毫无关联。然而,作为博士研究生,学术研究是我唯一的工作,除非我在学术研究上能有所成果,否则我将得不到博士学位。我难以把自己的情绪和每天的研究进展分离开来,而在那几个月中,我的研究进展出奇得缓慢。

我现在步入了一个完全陌生的领域,所以我很难再向本科时候那样向别人寻求帮助,因为问题的答案往往是不确定的。因为我是唯一试图将Klee应用于驱动程序代码的人,我的同事们并不能为我提供任何指导。Dawson偶尔会给我一些较高层面的策略性的建议,但就像所有已经获得终身职位的教授一样,他的角色并不是和他的学生们一起“在战壕里(第一线)参与战斗”。在做出研究成果的过程中,弄清楚所有复杂的细节是我们学生的任务——对我来说,这些细节就是如何找到别人从未发现过的、Linux驱动程序中的软件缺陷。教授们都喜欢重复这句老生常谈:“如果有人曾经做过这样的工作的话,那就不能叫做学术研究啦!”这是我第一次亲身体会到了这句话的含义。

尽管我每天都觉得工作很无望,但我仍然不断安慰自己:\emph{我只是刚刚开始在这里工作,我应该保持耐心。}我不想在我的导师或者同事面前显得软弱无能,尤其因为我当时是Dawson组里最年轻的学生。所以,我在超过100天的时间里,每天修复Klee不断产生的新问题,不断遇到新的、更加棘手的问题,在我的“寻找Linux驱动程序缺陷之旅”中,拖着沉重的脚步前进。

当时,在我醒着的每时每刻,我不是在工作,就是在思考研究中的问题,抑或是在因为自己在研究中被艰深的技术问题所困扰而感到沮丧。与一般的朝九晚五的工作(比如我的暑期实习)所不同的是,以前每天晚上我可以把工作留在办公室,坐在电视前放松自己;而现在的学术研究在情绪上和心理上都是无休无止的。我晚上几乎没法让大脑停止思考问题,尽情放松休息——后来我发现几乎所有的博士研究生都有类似的困扰。有时,因为我的研究任务

\end{CJK}
\end{document}